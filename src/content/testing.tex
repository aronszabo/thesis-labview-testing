%----------------------------------------------------------------------------
\chapter{Testing in LabVIEW}
%----------------------------------------------------------------------------


\section{Introduction to unit testing}
Unit Testing is a main part of the software development workflow. On the lowest software architecture level, unit tests verify only a single module or package of the code. The tested software runs isolated from other components, external dependencies substituted with custom testing code, so that running a test will not have any effect on the system. A test method usually tests a single functionality of the unit, has a set of inputs, and expects a set of outputs or some action to happen. There have been recent efforts to make unit test creation an automated process, since the simple structure of tests needs often only mechanical coding, instead of a creative process of software design. 
\section{Testing embedded systems}
Testing hardware prototypes has always been a difficult step of production, since it requires designing a test environment, which can be nearly as much effort as designing the product itself. Fortunately, there are hardware testing tools on the market, that can make this process rapid and efficient, providing solutions for the common tasks in testing. One of these toolkits is the set of hardware and software by National Instuments, which includes measurement and control modules, real-time and FPGA hardware, and a graphic programming environment, making prototyping easy.
\section{About LabVIEW}
LabVIEW is a system engineering software tool, developed by National Instruments, used commonly for rapid development of testing systems, measurement solutions, or other industrial applications. Its features include a graphic programming language, called G, which has advantages over a text-based programming language from a system engineering perspective. The language being graphic comes with a steep learning curve: the drag-and-drop and wiring methods of the software are really intuitive, and the language elements are provided in a toolbox. The program nodes are executed in a data flow, resulting that parallel execution is a natural feature of this language. This is taken advantage of in embedded systems, and it also makes LabVIEW a unique platform for testing, or perform symbolic execution on.
