%----------------------------------------------------------------------------
\chapter{Background}
%----------------------------------------------------------------------------
\section{Basics of Unit Testing}
\subsection{A brief introduction to software lifecycle models}
\subsection{Unit testing and testing patterns}
Unit Testing is a main part of the software development workflow. On the lowest software architecture level, unit tests verify only a single module or package of the code. The tested software runs isolated from other components, external dependencies substituted with custom testing code, so that running a test will not have any effect on the system. A test method usually tests a single functionality of the unit, has a set of inputs, and expects a set of outputs or some action to happen. There have been recent efforts to make unit test creation an automated process, since the simple structure of tests needs often only mechanical coding, instead of a creative process of software design. 
\subsection{Test automatization}
\section{Symbolic execution}

Symbolic execution \cite{King:1976:SEP:360248.360252}
\section{About LabVIEW}
LabVIEW is a system engineering software tool, developed by National Instruments, used commonly for rapid development of testing systems, measurement solutions, or other industrial applications. Its features include a graphic programming language, called G, which has advantages over a text-based programming language from a system engineering perspective. The language being graphic comes with a steep learning curve: the drag-and-drop and wiring methods of the software are really intuitive, and the language elements are provided in a toolbox. The program nodes are executed in a data flow, resulting that parallel execution is a natural feature of this language. This is taken advantage of in embedded systems, and it also makes LabVIEW a unique platform for testing, or perform symbolic execution on.
\subsection{Hardware and environment}
\subsection{The G programming language}