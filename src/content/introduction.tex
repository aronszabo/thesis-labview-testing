%----------------------------------------------------------------------------
\chapter{\bevezetes}
%----------------------------------------------------------------------------

Software on embedded and industrial systems are getting more popular and complex for the past few decades. Correct and reliable operation of these systems is a major concern, since failures can cause great damages, or even threathen human lives. Software testing is a method, which aims to investigate that the software works properly under most possible circumstances, reducing the risk of misbehaviour.
% all, prove = erős
The main topic of this thesis is software testing - with the focus on embedded systems, and methods to help developers automate the testing workflow. Since testing is present on all levels of software architecture, and most of the development phases, it should be done in an effective way, otherwise it will be very expensive. 

Testing hardware prototypes has always been a difficult step of production, since it requires designing a test environment, which can be nearly as much effort as designing the product itself. Fortunately, there are hardware testing tools on the market, that can make this process rapid and efficient, providing solutions for the common tasks in testing. One of these toolkits is the set of hardware and software by National Instuments, which includes measurement and control modules, real-time and FPGA hardware, and a graphic programming environment, making prototyping easy.

The hardware and software platform of National Instruments is a widely used solution for complex engineering tasks, like building control or measurement systems. One of its main advantages is quick prototyping: lots of common components, like signal processing or control loops are included and ready to use. The software components of this platform include LabVIEW, a programming environment, which has a unique, graphical, dataflow-driven programming language. The heart of the systems using the National Instruments platform are the programs written in LabVIEW: minimal wiring or electronic components are needed on the hardware side, everything is wired together in software.

Most testing methods, lots of libraries, and other helper software are available for traditional programming languages, like Java or C\#, but they are not so widespread in the world of LabVIEW. My aim is to investigate, what testing methods are already in use, how the testing efficiency can be improved, and implement a solution, which can help in the automation of test creation.
\section{Motivation}
Over the past year, I did my university projects in the field of software testing, and I learned about various testing techniques. One of them was automatic unit test generation, which in my opinion is not very effective on its own: generating tests with some inputs and exercising the output (which comes from an execution of the program) says very little. I also learned about mutation testing, a way to grade the quality of tests - the rating shows how sensitive the tests for modifications in the software are. In the case of automatic generation, poor quality tests can be filtered.

During summer internship I worked on a unit test framework for LabVIEW in a small team. I learned that very few testing tools exist for LabVIEW, and they are not advanced compared to similar tools for other platforms or programming languages. I think having an automatic test generation tool can help more advanced testing get in practice.

%A bevezető tartalmazza a diplomaterv-kiírás elemzését, történelmi előzményeit, a feladat indokoltságát (a motiváció leírását), az eddigi megoldásokat, és ennek tükrében a hallgató megoldásának összefoglalását.

%A bevezető szokás szerint a diplomaterv felépítésével záródik, azaz annak rövid leírásával, hogy melyik fejezet mivel foglalkozik.
\section{Results}
The main idea was to create a tool, that helps unit test creation with automatic test generation. In the beginning of the semester I did a resarch on the topic of unit testing and symbolic execution (a method to determine execution paths in a program), read articles and papers, and summarized them in the thesis. I designed a tool, that supports automatic test generation with providing input sets that cover maximal amount of source code. I implemented the tool with basic functionality, so that it could be later used to develop a full solution. I tested the finished tool on simple LabVIEW programs.
% nagy vonalakban mi keszult el
\section{Structure of the thesis}
% egy-egy mondat minden fejezetrol
The thesis breaks down to six main chapters. In the \textbf{Background} chapter, technologies and techniques used in the thesis or related to the topic will be introduced, providing an overall view on the topic. \textbf{Testing in LabVIEW} is about existing testing methods and frameworks, the solutions currently used in the industry. \textbf{Design} walks through the steps of planning the tool, defining the goals, requirements, and the architecture. The process of writing the software is in the \textbf{Implementation} chapter, along with the challenges I faced while developing the tool. Finally, in \textbf{Evaluation}, the tool is tested with demos that show the abilities of the finished prototype.
\pagebreak