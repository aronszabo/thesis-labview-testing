%----------------------------------------------------------------------------
\appendix
%----------------------------------------------------------------------------
\chapter*{\fuggelek}\addcontentsline{toc}{chapter}{\fuggelek}
\setcounter{chapter}{\appendixnumber}
%\setcounter{equation}{0} % a fofejezet-szamlalo az angol ABC 6. betuje (F) lesz
\numberwithin{equation}{section}
\numberwithin{figure}{section}
\numberwithin{lstlisting}{section}
%\numberwithin{tabular}{section}

%----------------------------------------------------------------------------
%\section{A TeXstudio felülete}
%----------------------------------------------------------------------------
%\begin{figure}[!ht]
%\centering
%\includegraphics[width=150mm, keepaspectratio]{figures/TeXstudio.png}
%\caption{A TeXstudio \LaTeX-szerkesztő.} 
%\end{figure}

%----------------------------------------------------------------------------
%\clearpage\section{Válasz az ,,Élet, a világmindenség, meg minden'' kérdésére}
%----------------------------------------------------------------------------
%A Pitagorasz-tételből levezetve
%\begin{align}
%c^2=a^2+b^2=42.
%\end{align}
%A Faraday-indukciós törvényből levezetve
%\begin{align}
%\rot E=-\frac{dB}{dt}\hspace{1cm}\longrightarrow \hspace{1cm}
%U_i=\oint\limits_\mathbf{L}{\mathbf{E}\mathbf{dl}}=-\frac{d}{dt}\int\limits_A{\mathbf{B}\mathbf{da}}=42.
%\end{align}


%installation, futtatas
%mi van a zpi mellekletben
%mellekletben tesztesetek

The source code of the symbolic execution tool and the tested VIs are included in the attachment of this BSc thesis. 

\paragraph{Contents of the attachment:}
\begin{itemize}
\item SymbolicTool: Source code of the tool
\item Test VIs

\begin{itemize}
\item Demo\textunderscore A.gvi: The example VI of Demo A
\item Demo\textunderscore B.gvi: The example VI of Demo B
\item Test\textunderscore C.gvi: VI with no case structures
\item Test\textunderscore D.gvi: A more complex example introducing mixed types of input data
\item Test\textunderscore E.gvi: An example on solving for multiplication
\item Test\textunderscore F.gvi: Testing for a 4-variable logical function

\end{itemize}
\end{itemize}
\section{Installation and run}

Before installation, make sure, the system meets the hardware requirements of LabVIEW NXG\footnote{\url{http://www.ni.com/hu-hu/shop/labview/compare-labview-nxg-and-labview.html}}.
\paragraph{Recommended software environment:}
\begin{itemize}
\item Microsoft Windows 7 / 8.1 / 10\footnote{\url{https://www.microsoft.com/en-us/windows}}
\item Microsoft .NET framework 4.6.2\footnote{\url{https://www.microsoft.com/en-us/download/details.aspx?id=53344}}
\item Microsoft Visual Studio 15\footnote{\url{https://visualstudio.microsoft.com/vs/}}
\item Microsoft Research Z3 Solver\footnote{\url{https://www.microsoft.com/en-us/download/details.aspx?id=52270}}
\item National Instruments LabVIEW NXG 2.1\footnote{\url{http://www.ni.com/en-us/shop/labview/labview-nxg.html}}
\end{itemize}
Open solution \textit{SymbolicTool.sln} with Visual Studio. After opening the project, the paths of referenced libraries might need to be adjusted to the paths on the actual system. 

After building, copy SymbolicTool.plugin.dll from the bin folder of the project to the Addons folder of LabVIEW NXG 2.1 (or a sub-folder). Copying Z3 libraries to the Addons folder might also be necessary. Run LabVIEW. A new entry called SymbolicTool should appear in the main menu bar. Select the commands of this menu, when a Block Diagram of a VI is active on the screen.
\begin{figure}
\centering
\includegraphics[width=150mm,keepaspectratio]{figures/interface.png}
\caption{Interface of LabVIEW NXG 2.1 with SymbolicTool installed} 
\label{fig:interface1}
\end{figure}

\paragraph{Debugging} Copy SymbolicTool.plugin.pdb to the Addons folder along with the DLL. Run LabVIEW and open a project. Select Debug / Attach to process... in Visual Studio and attach to LabVIEW NXG.exe. Now the debugger in Visual Studio can be used.
